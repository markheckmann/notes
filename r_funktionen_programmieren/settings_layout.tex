\usepackage{blindtext}
\usepackage{lmodern}
\usepackage[english, ngerman]{babel}
\usepackage{url} 
\usepackage{tikz}
\usetikzlibrary{shapes,positioning}
\usepackage{wallpaper}
\usepackage{xcolor} 
\usepackage{marginnote}
\usepackage[framemethod=TikZ]{mdframed}                  
\usepackage{enumitem}  
\usepackage{longtable}  
%%%%%%%%%%%%%%%%%%%%%%%%%%%%%%%%%%%%%%%%%%%%%%%%%%%%%%%%%%%%%%%%%%%%%%%%%%%%%%%
%%% FONT SETTING %%%
\usepackage[T1]{fontenc} 
\usepackage[utf8]{inputenc}             % nach hinten da sonst Konflikte mit Sweave
%\usepackage[applemac]{inputenc}
\DisemulatePackage{setspace}            % um setspace mit memoir zu nutzen
\usepackage[onehalfspacing]{setspace}   % singlespacing
\setstretch{1.0}                        % for other than standard spacings

%%%%%%%%%%%%%%%%%%%%%%%%%%%%%%%%%%%%%%%%%%%%%%%%%%%%%%%%%%%%%%%%%%%%%%%%%%%%%%%
%%% LAYOUT  
\usepackage{marvosym}
\usepackage{parskip}
%\usepackage[left=2.5cm, top=2cm, right=2.5cm, bottom=4cm, nohead, nofoot]{geometry} %for report

%\usepackage{fancyhdr}
\usepackage[centering, bindingoffset=1cm, headheight=15pt, includeheadfoot, textheight=24cm, textwidth=14cm]{geometry}
% Festlegung des Seitenstils (fancyhdr)  
%\usepackage{mdframed}

%\usepackage[usenames]{color}  % nut rausnehmen wenn driver HighlightWeaveDriver aus highlight package genutzt wird, da dieser auch usepackage{color} einfügt
\definecolor{titlepagecolor}{rgb}{0.060, 0.140, 0.430}
\definecolor{codecolor}{rgb}{0.400, 0.400, 0.400}
\definecolor{codecolor}{rgb}{.647,.129,.149}
\definecolor{uebungcolor}{rgb}{0.950, 0.950, 0.950}       
\definecolor{listingheadercolor}{rgb}{0.700, 0.700, 0.700}


%% roter Untersctrich am oberen Seitenrand für fancyhdr
\makeatletter
\def\headrule{{\color{nicered}\if@fancyplain\let\headrulewidth\plainheadrulewidth\fi
\hrule\@height\headrulewidth\@width\headwidth
\vskip-\headrulewidth}}
\makeatother

% \pagestyle{fancy}
% \fancyhf{}
% \fancyhead[LE,RO]{{\textcolor{nicered}\thepage}}
% %\fancyfoot[LO]{Titel der Arbeit}
% %\fancyfoot[RE]{Institut} 
% \fancyhead[LO]{\sffamily\nouppercase {\textcolor{nicered}{\rightmark}}}    %\slshape
% \fancyhead[RE]{\sffamily\nouppercase {\textcolor{nicered}{\leftmark}}}   

% \renewcommand{\headrulewidth}{0.5pt}
% %\renewcommand{\footrulewidth}{0.5pt}
% %\setlength{\headheight}{15pt}
% %\setlength{\footheight}{15pt}  

%\usepackage[]{geometry}  % for book
%\usepackage{a4wide}  
\setlength{\footskip}{1cm}   % Seitenzahl weiter n. unten, Achtung ggf. Problem bei Fussnoten
\setlength{\headsep}{1cm}
\setlength{\parindent}{0in}
\setlength{\parskip}{.1in}
\usepackage[hang]{footmisc} % flushmargin
\setlength{\footnotemargin}{.5cm}
\setlength{\footnotesep}{.5cm}
% \setlength{\textwidth}{160mm}
%\setlength{\oddsidemargin}{10mm}
%\usepackage[titletoc]{appendix}     % for named appendices
%\usepackage{hanging}            % for hanging indent environment
\clubpenalty = 100000    % Disable single lines at start of paragraph (Schusterjungen)
\widowpenalty = 100000   % Disable single lines at end of paragraph (Hurenkinder)
\displaywidowpenalty = 100000


%%%%%%%%%%%%%%%%%%%%%%%%%%%%%%%%%%%%%%%%%%%%%%%%%%%%%%%%%%%%%%%%%%%%%%%%%%%%%%%
%%% SUBJECT INDEX
\usepackage{makeidx}
\makeindex{}
\renewcommand{\indexname}{Subject index}      % change name of index page

%%%%%%%%%%%%%%%%%%%%%%%%%%%%%%%%%%%%%%%%%%%%%%%%%%%%%%%%%%%%%%%%%%%%%%%%%%%%%%%
%%% TABLE OF ... CONTENTS, FIGURES AND TABLES (cft)
%\usepackage{tocloft} 
%\settocdepth{subsection}       % subsection in toc bei memoir 
\setsecnumdepth{subsection}    % subsection nummerieren, macht memoir als default nicht
\renewcommand{\cftchapterfont}{\normalfont\sffamily\color{nicered}}       % chapter format
\renewcommand{\cftchapterformatpnum}{\normalfont\sffamily\color{nicered}} % chapter number 
\renewcommand*{\cftchapterafterpnum}{\par \vspace{2mm}}           % space after chapter    
\renewcommand{\cftsectionfont}{\normalfont\sffamily}              % section format  
\renewcommand{\cftsectionformatpnum}{\normalfont\sffamily}        % section number               
\renewcommand{\cftsubsectionfont}{\normalfont\sffamily}           % subsection format
\renewcommand{\cftsubsectionformatpnum}{\normalfont\sffamily}     % subsection number                   

\renewcommand{\cftfigurefont}{\normalfont\sffamily}       % figure format
\renewcommand{\cftfigureformatpnum}{\normalfont\sffamily} % figure number 
                                                                             
\renewcommand{\cfttablefont}{\normalfont\sffamily}       % table format
\renewcommand{\cfttableformatpnum}{\normalfont\sffamily} % table number

%%%%%%%%%%%%%%%%%%%%%%%%%%%%%%%%%%%%%%%%%%%%%%%%%%%%%%%%%%%%%%%%%%%%%%%%%%%%%%%
%%% FIGURE AND TABLE SETTINGS 
\usepackage{booktabs}
\usepackage[flushleft]{threeparttable}
\usepackage{tabularx}
%\usepackage[tight]{subfigure} 
\usepackage{subfig}  
\usepackage{rotating}
\usepackage{float}
%\restylefloat{figure}     % H will mean directly here now .
                           % funktioniert nicht zusammen mit memoir!!!
\usepackage[]{caption}   %bold caption label
\captionsetup[figure]{font=footnotesize, labelfont={footnotesize, color=nicered, bf}, labelsep=space}                                  % labelsep=newline
\captionsetup[table]{font=footnotesize, labelfont={footnotesize, bf}, labelsep=space}
\newcommand\T{\rule{0pt}{2.6ex}}          % global change toprule spaces table
\newcommand\B{\rule[-1.2ex]{0pt}{0pt}}    % global change bottomrule table
% \setlength{\subfigtopskip}{5mm}           % set skips for subfigures and tables
% \setlength{\subfigcapskip}{1mm}
%  
\usepackage{ragged2e}
\newcolumntype{x}[1]{>{\hsize=#1\hsize}X}
\newcolumntype{y}[1]{>{\RaggedRight\arraybackslash\hsize=#1\hsize}X}

%%%%%%%%%%%%%%%%%%%%%%%%%%%%%%%%%%%%%%%%%%%%%%%%%%%%%%%%%%%%%%%%%%%%%%%%%%%%%%%
\usepackage{pstricks}  % für ovale boxen
\usepackage{xparse}    % für NewDocumentCommand

%%%%%%%%%%%%%%%%%%%%%%%%%%%%%%%%%%%%%%%%%%%%%%%%%%%%%%%%%%%%%%%%%%%%%%%%%%%%%%%
%%% HYPHENATIONS %%%
%\hyphenation{OpenRepGrid}       % no hyphenation in these words
%%%%%%%%%%%%%%%%%%%%%%%%%%%%%%%%%%%%%%%%%%%%%%%%%%%%%%%%%%%%%%%%%%%%%%%%%%%%%%%
%%% COMMANDS 
\title{My Title}
\author{Mark Heckmann}
\newcommand{\code}[1]{\texttt{#1}}
\newcommand{\pkg}[1]{{\normalfont\fontseries{b}\selectfont #1}}
\newcommand{\R}{{\sffamily R}}
\newcommand{\rtextsize}{{\tiny}}
\newcommand{\webinfo}[2]{
  \marginpar{
    \href{#1}{
    \includegraphics[width=8mm]{pics/infosymbol.jpg}}
    \scriptsize \textit{#2}
}}
\newcommand{\webinfosmall}[1]{  
    \href{#1}{\textsuperscript{\Info}}
}                     
% switch für schwarz-weiß Grafiken. Jede Grafik hat eine SQ Version mit dem Zusatz _bw
\newcommand{\bw}{_bw} % {}  

%%%%%%%%%%%%%%%%%%%%%%%%%%
% memoir definition of style, raus, wenn kein memoir genutzt werden soll

\usepackage{calc, graphicx, soul, fourier}
\definecolor{nicered}{rgb}{.647,.129,.149}
\makeatletter
\newlength\dlf@normtxtw
\setlength\dlf@normtxtw{\textwidth}
\def\myhelvetfont{\def\sfdefault{mdput}}
\newsavebox{\feline@chapter}
\newcommand\feline@chapter@marker[1][4cm]{%
  \sbox\feline@chapter{%
  \resizebox{!}{#1}{\fboxsep=1pt%
  \colorbox{nicered}{\color{white}\bfseries\sffamily\thechapter}%
  }}%
  \rotatebox{90}{%
  \resizebox{%
    \heightof{\usebox{\feline@chapter}}+\depthof{\usebox{\feline@chapter}}}%
  {!}{\scshape\so\@chapapp}}\quad%
    \raisebox{\depthof{\usebox{\feline@chapter}}}{\usebox{\feline@chapter}}%
}
\newcommand\feline@chm[1][4cm]{%
  \sbox\feline@chapter{\feline@chapter@marker[#1]}%
  \makebox[0pt][l]{% aka \rlap
  \makebox[1cm][r]{\usebox\feline@chapter}%
}}
\makechapterstyle{daleif1}{
\renewcommand\chapnamefont{\normalfont\Large\scshape\raggedleft\so\sffamily}
\renewcommand\chaptitlefont{\normalfont\huge\scshape\color{nicered}\sffamily}%\bfseries
\renewcommand\chapternamenum{}
\renewcommand\printchaptername{}
\renewcommand\printchapternum{\null\hfill\feline@chm[2.5cm]\par}
\renewcommand\afterchapternum{\par\vskip\midchapskip}
\renewcommand\printchaptertitle[1]{\chaptitlefont\raggedleft ##1\par}
}
\makeatother  
\chapterstyle{daleif1}  

% change section looks other color
\usepackage{titlesec}     % to change vertical space before chapter
% \titleformat{\chapter}[display]
% {\normalfont\huge\bfseries}{\chaptertitlename\ \thechapter}{20pt}{\Huge}
% % this alters "before" spacing (the second length argument) to 0
% \titlespacing*{\chapter}{0pt}{0pt}{40pt}

\titleformat{\section}    % Farbe der Überschriften ändern
{\color{nicered}\normalfont\Large\sffamily}  %\bfseries
{\color{nicered}\thesection}{1em}{}

\titleformat{\subsection}    % Farbe der Überschriften ändern
{\color{nicered}\normalfont\large\sffamily}  %\bfseries
{\color{nicered}\thesubsection}{1em}{}   
                  
                  
%%%%%%%%%%%%%%%%%%%%%%%%%%
\usepackage{listings}     % für Beispiel R Code
% \DeclareCaptionFont{white}{\color{white}}
% \DeclareCaptionFormat{listing}{\colorbox{listingheadercolor}{\parbox{\textwidth}{#1#2#3}}}
% \captionsetup[lstlisting]{format=listing, labelfont=white, textfont=white}
% \lstset{basicstyle=\scriptsize \ttfamily, 
%         frame=blr, rulesepcolor=\color{gray},
%         columns=fixed,
%         literate={ö}{{\"o}}1
%                  {ä}{{\"a}}1
%                  {ü}{{\"u}}1}   

\DeclareCaptionFont{fontlist}{\bfseries \sffamily\color{nicered}}
\DeclareCaptionFormat{listingformat}{#1#2#3\par\vspace{4mm}}
%\DeclareCaptionLabelFormat{listing}{(\bfseries \sffamily {#2})}
\captionsetup[lstlisting]{format=listingformat, labelfont=fontlist, textfont=fontlist,
                          font={bf, footnotesize}}
\lstset{basicstyle=\scriptsize \ttfamily, 
      frame=none, 
      columns=fixed,
      xleftmargin=2mm,
      framesep=2mm, 
      framerule=1pt, 
      rulecolor=\color{nicered},
      fillcolor=\color{uebungcolor},
      framextopmargin=5mm,
      framexbottommargin=3mm,
      breaklines=f, 
      literate={ö}{{\"o}}1
               {ä}{{\"a}}1
               {ü}{{\"u}}1} 
                                 

% colored fbox
\definecolor{fboxcolor}{rgb}{0.500, 0.500, 0.500} 
\newcommand{\picfbox}[1]{{\color{fboxcolor}\fbox{\normalcolor#1}}}

%\newcommand\picfbox{\fbox}

%%%%%%%%%%%%%%%%%%%%%%%
%%%%  Seitenboxen  %%%%
%%%%%%%%%%%%%%%%%%%%%%%  

% DOES NOT WORK WELL!
\NewDocumentCommand\videosym{O{}m}{%
\noindent \marginpar{ 
\hspace{2em}%
\includegraphics[width=1cm]{pics/symbol_video.pdf} %
} %
}   
               

%%%%%%%%%%%%%%%%%%%%%%%
%%%% Übungsboxen 1 %%%%
%%%%%%%%%%%%%%%%%%%%%%%   
                
% \marginpar{\centering{\includegraphics[width=2cm]{pics/nerd.png}}}
% Übungsumgebung definieren
\makeatletter
\newenvironment{uebung}
{ % image on side
\begin{lrbox}{\@tempboxa}
  \begin{minipage}{\columnwidth}}
{\end{minipage}\end{lrbox}%
   \colorbox{uebungcolor}{\usebox{\@tempboxa}}} 
\makeatother

\newcounter{uebung_no}
\newenvironment{uebung_item}
   {\stepcounter{uebung_no}
   \begin{enumerate}\item[\arabic{uebung_no}.] \color{black}}
   {\end{enumerate}}   
       
          
% weitere Boxen für Übungen
% \usepackage{fancybox}
% \definecolor{white}{rgb}{1.000, 1.000, 1.000}
% \definecolor{darkgray}{rgb}{0.900, 0.900, 0.900}
% \definecolor{gray}{rgb}{0.500, 0.500, 0.500}
% 
% \newenvironment{infobox}[1][Übungen]{
% \begin{center}
% \begin{Sbox}
% \begin{minipage}{.9\textwidth}
% \textcolor{white}{\large\textsc{#1}}
% \end{minipage}
% \end{Sbox}
% \colorbox{darkgray}{\TheSbox}
% \begin{Sbox}
% \begin{minipage}{.9\textwidth}
% }
% {
% \end{minipage}
% \end{Sbox}
% \colorbox{gray}{\TheSbox}
% \end{center}
% }
% %%%%%%
% \begin{infobox}{Test}
% Some info to be put here  
% \end{infobox}  

\usepackage{framed}

\newcounter{uebungsbox_no}
\newenvironment{uebungsbox}{%
  \def\FrameCommand{\fboxsep=\FrameSep \fcolorbox{uebungcolor}{uebungcolor}}%
  \color{black} \MakeFramed {\FrameRestore}%
  \stepcounter{uebungsbox_no}}%
 {\endMakeFramed}

\usepackage[hidelinks]{hyperref}   % mactex muss geupdated werden um hidelinks nutzen zu können, da neues argument, solange folgendes:
\hypersetup{
   colorlinks=false,
   pdfborder={0 0 0},
}       

%%%%%%%%%%%%%%%%%%%%%%%
%%%% Übungsboxen 2 %%%%
%%%%%%%%%%%%%%%%%%%%%%%
% guter Ansatz mit mdframed, macht jedoch Fehler beim Umbruch und ich
% kann den Grund nicht finden. So lange muss ich weiter
% den oberen Ansatz nutzen (außer der lernziel box, die wird schon genutzt)  

\definecolor{myred}{rgb}{.647,.129,.149}
\definecolor{background}{rgb}{.980, .980, .980}

\newcounter{uebung_2}
\newenvironment{uebung_2}
   {\stepcounter{uebung_2}
      \begin{enumerate}\item[\arabic{uebung_2}.] \color{black}}
   {\end{enumerate}}

\newcounter{uebungsbox_2}
\newenvironment{uebungsbox_2}
   {\stepcounter{uebungsbox_2} 
      \begin{mdframed}[style=uebungsboxstyle]
      {\bfseries \color{myred} Übungsbox \arabic{uebungsbox_2}} \par}
   {\end{mdframed}} 

     
% \newenvironment{lernzielbox_alt}
%  {\begin{mdframed}[style=uebungsboxstyle, 
%                    skipbelow=\baselineskip] }
%  {\end{mdframed}}  

% Box am Kapitelanfang, in der die Lernziele stehen.
% Muss mit items (\item Zeilen) gefüllt werden
\newenvironment{lernzielbox}[1][Was bietet dieses Kapitel?]
  { \begin{center}
    \begin{minipage}{.9\linewidth}
    \begin{mdframed}[style=uebungsboxstyle, 
                     skipbelow=\baselineskip]
    {\bfseries \sffamily \color{nicered} #1} \par
    \begin{enumerate}[itemsep=1pt, topsep=12pt, partopsep=0pt, 
                      label=\color{nicered}\theenumi] }
  { \end{enumerate} 
    \end{mdframed}
    \end{minipage} 
    \end{center}  }    

% Box am Kapitelanfang, in der die Lernziele stehen.
% Muss mit items (\item Zeilen) gefüllt werden
\newenvironment{literaturverweise}[1][Weiterführende Literatur]
  { \begin{center}
    \begin{minipage}{.95\linewidth}
    \begin{mdframed}[style=uebungsboxstyle, 
                     skipbelow=\baselineskip]
    {\bfseries \sffamily \color{nicered} #1} \par }
  { \end{mdframed}
    \end{minipage} 
    \end{center}  }  
        
\newenvironment{codebox}[1][]
  { \begin{center}
    %\begin{minipage}{1.0\textwidth}
    \begin{mdframed}[style=uebungsboxstyle, 
                     skipbelow=\baselineskip]
    {\bfseries \sffamily \color{nicered} #1} \par }
  { \end{mdframed}
   % \end{minipage} 
    \end{center}  }  
                             
\global\mdfdefinestyle{uebungsboxstyle}{%
  linecolor=myred, roundcorner=10pt,
  backgroundcolor=background, middlelinewidth=1pt,%
  leftmargin=1cm, rightmargin=1cm
}


%%%%%%%%%%%%%%
%%% Footer %%%
%%%%%%%%%%%%%%

%% Header and footer heights
\setheadfoot{\baselineskip}{10mm}
\setlength\headsep{7mm}

\newlength\pagenumwidth
\settowidth{\pagenumwidth}{99}

%% Define style of page number colour box
\tikzset{pagefooter/.style={
anchor=base, font=\sffamily\bfseries\small,
text=white, fill=nicered!80!black, text centered,
text depth=17mm,text width={\pagenumwidth + 1mm}}}


%%%%%%%%%%
%%% Re-define running headers on non-chapter odd pages
%%%%%%%%%%
\makeoddhead{headings}
%% Left header is empty but I'm using it as a hook to paint the
%% background rectangles underneath everything else  MH: MilkTea!25!white to white!25!white   
{\begin{tikzpicture}[remember picture,overlay]
\fill[white!25!white] (current page.north east) 
  rectangle (current page.south west);
\fill[white, rounded corners] 
  ([xshift=-10mm,yshift=-20mm]current page.north east) rectangle 	
	([xshift=15mm,yshift=17mm]current page.south west);
\end{tikzpicture}}%
%% Blank centre header
{}%
%% Display a decorate line and the right mark (chapter title)
%% at right end
{\begin{tikzpicture}[xshift=-.75\baselineskip,yshift=.25\baselineskip,remember picture, overlay,fill=nicered,draw=nicered]\fill circle(3pt);\draw[semithick](0,0) -- (current page.west |- 0,0);\end{tikzpicture}\sffamily\itshape\small\rightmark}

%%%%%%%%%%
%%% Re-define running footers on odd pages
%%% i.e. display the page number on the right
%%%%%%%%%%
\makeoddfoot{headings}{}{}{%
\tikz[baseline]\node[pagefooter]{\thepage};}
\makeoddfoot{plain}{}{}{\tikz[baseline]\node[pagefooter]{\thepage};}


%%%%%%%%%%
%%% Re-define running headers on non-chapter even pages
%%%%%%%%%%
\makeevenhead{headings}
%% Draw the background rectangles; then the left mark (section
%% title) and the decorate line  MH: MilkTea!25!white to white!25!white
{{\begin{tikzpicture}[remember picture,overlay]
\fill[white!25!white] (current page.north east) rectangle (current page.south west);
\fill[white, rounded corners] ([xshift=-15mm,yshift=-20mm]current page.north east) rectangle ([xshift=10mm,yshift=17mm]current page.south west);
\end{tikzpicture}}%
\sffamily\itshape\small\leftmark\ 
\begin{tikzpicture}[xshift=.5\baselineskip,yshift=.25\baselineskip,remember picture, overlay,fill=nicered,draw=nicered]\fill (0,0) circle (3pt); \draw[semithick](0,0) -- (current page.east |- 0,0 );\end{tikzpicture}}{}{}
\makeevenfoot{headings}{\tikz[baseline]\node[pagefooter]{\thepage};}{}{}
\makeevenfoot{plain}{\tikz[baseline]\node[pagefooter]{\thepage};}
%% Empty centre and right headers on even pages
{}{}         

%%%%%%%%%%%%%%%%
%%% FOOTNOTE %%%
%%%%%%%%%%%%%%%%
% standard definition from memoir manual (p. 248) plus colors
\renewcommand*{\footnoterule}{%
	\color{nicered}
	\kern-3pt%
	\hrule width 0.4\columnwidth
	\kern 2.6pt
	\color{black}}

	
%%%%%%%%%%%%%%%%%%%%%%%%%%%%%%%%%%%%%%%%%%%%%%%%%%%%%%%%%%%%%%%%%%%%%%%%%%%%%%%

\usepackage{apager}             
\bibliographystyle{apager_dgps}
 