                                               %%%%%%%%%%%%%%%%%%%%%%%%%%%%%%%%%%%%%%%%%%%%%%%%%%%%%%%%%%%%%%%%%%%%%%%%%%%%%%%
%%% SWEAVE
       % Ihaka, R. (2009). Customizing Sweave 
%%%%%%%%%%%%%%%%%%%%%%%%%%%%%%%%%%%%%%%%%%%%%%%%%%%%%%%%%%%%%%%%%%%%%%%%%%%%%%%
%%% CUSTOMIZING SWEAVE 
%%% from: Ihaka, R. (2009). Customizing Sweave to Produce Better Looking LATEX Output
\DefineVerbatimEnvironment{Sinput}{Verbatim}{fontsize=\footnotesize, formatcom=\color{codecolor}, xleftmargin=2em}
\DefineVerbatimEnvironment{Soutput}{Verbatim}{fontsize=\footnotesize, xleftmargin=2em, formatcom=\color{codecolor}} 
\DefineVerbatimEnvironment{Scode}{Verbatim}{fontsize=\footnotesize, xleftmargin=2em, formatcom=\color{codecolor}}

\renewenvironment{Schunk}{\vspace{10pt}}{\vspace{8pt}}   
%%%%%%%%%%%%%%%%%%%%%%%%%%%%%%%%%%%%%%%%%%%%%%%%%%%%%%%%%%%%%%%%%%%%%%%%%%%%%%%
%%%%%%%%%%%%%%%%%%%%%%%%%%%%%%%%%%%%%%%%%%%%%%%%%%%%%%%%%%%%%%%%%%%%%%%%%%%%%%%


\section{Varianzschätzung}

Die Definition der Varianz einer Zufallsvariable $X$ ist wohlbekannt. 

\begin{equation}
\operatorname{Var}(X) = \operatorname{E}\left( (X-\mu)^2\right)  
\end{equation}  

Sie berechnet sich wie folgt, wenn der wahre Mittelwert $\mu$ der Variablen $X$ bekannt ist. 

\begin{equation} \label{eq:var_mu}
  \sigma^2 = \frac{\sum{(x_i - \mu)^2}}{n}  
\end{equation}

Dies ist jedoch in der Regel nicht der Fall, so dass $\bar x$ geschätzt werden muss. So könnte man $\mu$ wird dann durch diesen Schätzwert ersetzen.

\begin{equation} \label{eq:var_mean}
  S^2 = \frac{\sum{(x_i - \bar x)^2}}{n}  
\end{equation} 

Hierbei handelt es sich jedoch nicht mehr um eine erwartungstreue Schätzung der Varianz von $X$. Dass dies nicht der Fall ist, soll nachfolgend simuliert werden. Hierzu programmieren wir zwei Funktionen. Eine die den Populationsmittelwert (Formel \ref{eq:var_mu}) und eine die den Stichprobenmittelwert(Formel \ref{eq:var_mean}) zur Schätzung der Varianz nutzt.  


\begin{Schunk}
\begin{Sinput}
> var_mu <- function(x, mu){   
   sum((x - mu)^2) / length(x)  
 }   
> var_mean <- function(x){
   xm <- mean(x)
   sum((x - xm)^2) / length(x)
 }
> x <- rnorm(1e3, 100, 10)
> var_mu(x, 100)  
\end{Sinput}
\begin{Soutput}
[1] 99.52682
\end{Soutput}
\begin{Sinput}
> var_mean(x)
\end{Sinput}
\begin{Soutput}
[1] 99.50176
\end{Soutput}
\end{Schunk}

Wir sehen, dass zwischen den beiden Schätzern Unterschiede bestehen können. An diesem Punkt ist es jedoch schwer zu sagen, ob diese Unterschiede einen Einfluss auf die Güte der Schätzung haben könnten. Aus diesem Grund simulieren wir viele Stichprobenziehungen.

\begin{Schunk}
\begin{Sinput}
> compare_est <- function(nrep, n, mu, s){
   res.mu <- rep(NA, nrep)         # Ergebnisvektor mu
   res.mean <- rep(NA, nrep)       # Ergebnisvektor mean
   counter <- 1                    # Zähler
   for (i in 1:nrep){
     cat("\r run", i)              # Ausgabe Durchlauf
     flush.console()               # Zwischenspeicher entleeren
     x <- rnorm(n, mu, s)          # Stichprobe ziehen
     res.mu[i] <- var_mu(x, mu)    # Varianzschätzung mu
     res.mean[i] <- var_mean(x)    # Varianzschätzung mean
     counter <- counter + 1        # Zähler erhöhen 
   }
   c(var.mu= mean(res.mu),         # mu und mean Schätzung
     var.mean=mean(res.mean))      # ausgeben
 }
> compare_est(1e3, 100, 100, 10)
\end{Sinput}
\end{Schunk}

\begin{Schunk}
\begin{Soutput}
   var.mu  var.mean 
100.17497  99.16071 
\end{Soutput}
\end{Schunk}

Um eine besser Aussage treffen zu können simulieren wir die Daten erneut für verschiedene Stichprobenumfänge $n$.

\begin{Schunk}
\begin{Sinput}
> ns <- seq(10, 100, 10)              # verschiedene n
> len.ns <- length(ns)                # Anzahl an n Werten
> rmat <- matrix(NA, len.ns, 2)       # Ergebnismatrix initialisieren
> for (i in 1:len.ns)
   rmat[i, ] <- compare_est(1e3, ns[i], 100, 10)    
> r <- as.data.frame(rmat)            # in dataframe verwandeln
> names(r) <- c("var.mu", "var.mean") # Spalten benennen
> r <- cbind(n=ns, r)                 # n Spalte hinzufügen
> r
\end{Sinput}
\begin{Soutput}
     n    var.mu var.mean
1   10 100.06657 89.64241
2   20  98.76847 94.02082
3   30  99.60366 96.19998
4   40  99.52436 96.90844
5   50  99.72247 97.70548
6   60 100.09744 98.45187
7   70 100.02446 98.57086
8   80  99.15451 97.92934
9   90  99.76741 98.58419
10 100 100.86990 99.94020
\end{Soutput}
\end{Schunk}

Es wird deutlich, dass die Schätzfunktion auf Basis der Mittelwertes die Varianz systematisch unterschätzt. Formel \ref{eq:var_mean} liefert somit keine erwartungstreue Schätzung von $\sigma^2$. Mit steigendem $n$ näheren sich die Schätzer jedoch an.  Welcher Zusammenhang besteht hierbei zwischen den beiden Schätzern? 

\begin{Schunk}
\begin{Sinput}
> r <- transform(r, diff=var.mu - var.mean)
> r <- transform(r, n_diff=n * diff) 
> r
\end{Sinput}
\begin{Soutput}
     n    var.mu var.mean       diff    n_diff
1   10 100.06657 89.64241 10.4241538 104.24154
2   20  98.76847 94.02082  4.7476497  94.95299
3   30  99.60366 96.19998  3.4036721 102.11016
4   40  99.52436 96.90844  2.6159222 104.63689
5   50  99.72247 97.70548  2.0169902 100.84951
6   60 100.09744 98.45187  1.6455749  98.73450
7   70 100.02446 98.57086  1.4536003 101.75202
8   80  99.15451 97.92934  1.2251662  98.01329
9   90  99.76741 98.58419  1.1832200 106.48980
10 100 100.86990 99.94020  0.9296927  92.96927
\end{Soutput}
\end{Schunk}

Die Differenz zwischen den Schätzfunktionen scheint systematisch mit $n$ zusammenzuhängen.

Um aus Formel \ref{eq:var_mean} einen erwartungstreuen Schätzer zu machen muss diese um eine Korrekturfaktor $\frac{n}{n-1}$ erweitert werden. 

\begin{Schunk}
\begin{Sinput}
> transform(r, corr.var.mean=var.mean *n / (n-1))
\end{Sinput}
\begin{Soutput}
     n    var.mu var.mean       diff    n_diff corr.var.mean
1   10 100.06657 89.64241 10.4241538 104.24154      99.60268
2   20  98.76847 94.02082  4.7476497  94.95299      98.96928
3   30  99.60366 96.19998  3.4036721 102.11016      99.51723
4   40  99.52436 96.90844  2.6159222 104.63689      99.39327
5   50  99.72247 97.70548  2.0169902 100.84951      99.69946
6   60 100.09744 98.45187  1.6455749  98.73450     100.12054
7   70 100.02446 98.57086  1.4536003 101.75202      99.99942
8   80  99.15451 97.92934  1.2251662  98.01329      99.16895
9   90  99.76741 98.58419  1.1832200 106.48980      99.69188
10 100 100.86990 99.94020  0.9296927  92.96927     100.94970
\end{Soutput}
\end{Schunk}

Die erwartungstreue Varianzschätzung auf Basis einer Stichprobe ist somit

\begin{eqnarray} \label{eq:var_sample}
  S^2 &=& \frac{n}{n-1} \frac{\sum{(x_i - \bar x)^2}}{n} \\  
           &=& \frac{\sum{(x_i - \bar x)^2}}{n-1}.
\end{eqnarray}

\par
\textbf{Berechnung mit Funktionen aus der \texttt{apply}-Familie}  

Alternativ zu eienr Schliefe können auch Funktionen der \texttt{apply}-Familie genutzt werden. 

\begin{Schunk}
\begin{Sinput}
> compare_est_2 <- function(reps, n, mu, s){
   x <- replicate(reps, rnorm(n, mu, s))
   vars_mu <- apply(x, 2, var_mu, mu)
   vars_mean <- apply(x, 2, var_mean)
   c(var.mu= mean(vars_mu),              # mu und mean Schätzung
     var.mean=mean(vars_mean))           # ausgeben  
 }
> compare_est_2(1e3, 100, 100, 10)
\end{Sinput}
\begin{Soutput}
   var.mu  var.mean 
100.03783  99.14195 
\end{Soutput}
\begin{Sinput}
> ns <- seq(10, 100, 10)
> res <- mapply(compare_est_2, reps=1e3, n=ns, mu=100, s=10)
> r <- as.data.frame(t(res))              # in dataframe verwandeln
> r <- cbind(n=ns, r)                     # n Spalte hinzufügen
> r
\end{Sinput}
\begin{Soutput}
     n    var.mu  var.mean
1   10 100.34796  90.62385
2   20  99.49068  94.46534
3   30 100.57160  97.18186
4   40 100.09461  97.59578
5   50 100.57407  98.45921
6   60  99.39860  97.71116
7   70 100.14886  98.77468
8   80 101.30143 100.00589
9   90  99.82390  98.69558
10 100  99.73707  98.80649
\end{Soutput}
\end{Schunk}


\textbf{Die analytische Herleitung der korrigierten Stichprobenvarianz}

\begin{align} 
    E (S_1^2)  &= \frac 1n \sum_{i=1}^n E\left( (X_i-\overline{X})^2 \right)= 
  \frac 1n E \left(\sum_{i=1}^n (X_i-\mu+\mu-\overline{X})^2\right)\\
  &=\frac1n E \left(\sum_{i=1}^n \left((X_i-\mu)^2 - 2(X_i-\mu)
  (\overline{X}-\mu) + (\overline{X}-\mu)^2\right) \right)\\
  &=\frac1n E\left(\sum_{i=1}^n (X_i-\mu)^2 - 2\sum_{i=1}^n(X_i-\mu)
  (\overline{X}-\mu) + n(\overline{X}-\mu)^2\right) \\
  &=\frac1n E\left(\sum_{i=1}^n (X_i-\mu)^2 - 2n(\overline{X}-\mu)
  (\overline{X}-\mu) + n(\overline{X}-\mu)^2\right) \\
  &=\frac1n E\left(\sum_{i=1}^n (X_i-\mu)^2 -n(\overline{X}-\mu)^2\right) \\
  &=\frac1n \left(\sum_{i=1}^n E\left((X_i-\mu)^2\right) -nE\left((\overline{X}-\mu)^2\right)\right) \\
  &=\frac1n \left(nVar(X)-nVar(\overline{X})\right) \\
  &=Var(X)-Var(\overline{X}) = \sigma^2 - \frac{\sigma^2}{n} = \frac{n-1}{n}\,\sigma^2, \end{align} 


 




%%%%%%%%%%%%%%%%%%%%%%%%%%%%%%%%%%%%%%%%%%%%%%%%%%%%%%%%%%%%%%%%%%%%%%%%%%%%%%%
